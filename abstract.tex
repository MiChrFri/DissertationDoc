\section*{Abstract}


Writing Software requires logical thinking, creativity, problem solving skills and of course teamwork. The whole process requires a good level of cognitive performance and a solid communication within a team.\\
However, this performance is not constant, at least for the most people. There is always a risk, a developer is producing bad code which could lead to expensive bugs and/or delays. 
Many developers don't really know about the quality of their code, neither in a general perspective nor in their temporary performance. Even if they would know, it would be not always be obvious to find reasons for negative or positive changes in their code quality.
Software metrics are around for decades with the purpose to evaluate the quality and the performance of the programmer but they are used for project management rather than for providing feedback to the developers. 
\bigbreak
This dissertation investigates correlations between external influences and the coding quality and the cognitive performance of programmers. 
In two experiments, mobile devices are being used to collect the contextual data of the environment and the behavior of the programmers. 
The installed application on the device of the participants gathers data from the device. It accesses the light sensor, the amplitude of the microphone, the step counter, a 3axis-accelerometer and the location of the device. These information are then clustered and linked to a context. 
The first experiment investigates the influences compared to other participants in order to find general factors. In the second experiment, the focus lays on individual information which are being created by a single participant. 
\bigbreak
The result of the individual experiment showed evidence of a negative impact of caffeine in the cognition of the participant and influences of different music. The participants performed a given task fast while listening to classical music compared to no music at all and the even worse resulting heavy metal. 
The experiment with a group of different participants didn't provide as much evidence as in the single experiment. 


