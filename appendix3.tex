\chapter{Programming task}

The programming task for the crowd Experiment.


\section{Palindromes}

\subsection{Question}
Generate a palindrome with the maximum possible amount of characters from an input string.\\
Count the amount of characters from the input that you didn't use in your palindrome then add 65 to the result and convert the result to the respresending ASCII character

\subsubsection{What's a palindrome?}
A palindrome is a word which reads the same from left to right and right to left such as anna or racecar

\subsubsection{Input structure}
a String with characters from a-z and whitespace\\
The first line indicates the number of test cases\\
The following lines are the individual test cases\\

\subsection{Example}
\subsubsection{input}
2\\
hello my world\\
amazing code\\

\subsubsection{Explanation Testcase 1}
The largest palindrome you can create from the characters:\\
hello my world\\
\quad \textbf{lohol}\\
These are 7 leftover characters:\\
\quad emywrld \quad\quad\quad\quad\quad\quad\quad\quad\textbackslash\textbackslash char.count = 7\\
When we add 65 we get 72, which is a 'H' in the ASCII table\\
65 + 7 = 72 \quad\quad\quad\quad\quad\quad\textbackslash\textbackslash char:'H'\\

\subsubsection{Explanation Testcase 2}
The largest palindrome you can create from the characters:\\
amazing code\\
\quad \textbf{ama}\\
These are 8 leftover characters:\\
\quad \textbf{zingcode}\quad\quad\quad\quad\quad\quad\textbackslash\textbackslash char.count = 8\\
When we add 65 we get 73, which is a 'I' in the ASCII table\\
65 + 8 = 73\quad\quad\quad\quad\quad\textbackslash\textbackslash char: 'I'

\textbf{Result}\\
HI
