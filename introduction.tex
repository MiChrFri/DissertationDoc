\chapter{Introduction}

\begin{quote}
\centering 
\em % optional -- to switch to emphasis (italics) mode
"Be a yardstick of quality. Some people aren't used to an environment where excellence is expected."

\medskip
\raggedleft
Steve Jobs
\end{quote}
\vspace{10 mm}

A computer is the powerful hardware that is able to do much more than a human being is capable of. However, The enormous capacity is useless without software which is making use of its power. In order to make the increasing computations controllable and keep the machines as customizable as possible, different layers of abstraction actually make it possible to manage the computing power. The abstraction reaches from machine readable code over Assembly to high level programming languages which can almost be read as an English sentence. On top of that, there are libraries and frameworks that provide features that already have been implemented \cite{Martin:2008:CCH:1388398}. A big part of the complexity is already encapsulated and the software engineers can focus on the functionality of the software and their specific problems.
\bigbreak
Over the years, the performance of the computers rapidly increased and with it the complexity of the code \cite{wirth2008brief}.
Software can be a simple tool that is written in a short time by a single person or it can be a gigantic software project with several hundred developers working on it\cite{cusumano1997microsoft}.
In order to allow to split the work on a software project, an encapsulation of the modules is mandatory. A general structure must be given to ensure that the different parts can integrate hand in hand and to keep the code understandable.\\
The more people work on one project, the more important it is to provide an organized and well planned architecture to keep the code clean.
\bigbreak
In today's world, software is everywhere; the traffic is controlled by computers as well as security systems, nuclear power plants or just a messenger app on a mobile phone etc. \\
The ubiquity of computers can make life easier, but can also cause unpredictable trouble.\\
In the early 1890s at the United Kingdom's Royal Air Force, an engineer found a bug that could have fired a missile without any command. Luckily it was found before a disaster happened \cite{ross2005exterminators}.
\bigbreak
The quality requirements varies for different software products. A crashing weather app on a mobile is not as bad as a bug that is causing a production stop in a plant. 
However, the quality of the software can make the difference whether a company will be successful or just be one of many abortive start-ups with a good idea but a bad execution.
\\
In the software industry, the most significant factor in the creating process is the human. The quality strongly depends on the performance of the programmers as a single person or in a team. That performance quality can change by various different reasons even a few times a day.
\bigbreak 
Previous researchers already did a lot of work and research in that area. \\
I will start the dissertation by summarizing their findings which will include previous studies that investigated different theories about influencing factors on cognitive performance and related work.

\bigbreak
In this dissertation, I will write about my approach where I used the sensors and information provided by the user's mobile phone to find evidence in factors that influence software quality. 
I developed an Android application for being installed on the participants mobile device. The app is gathering the location of the user, collecting sensor data from the light sensor, accelerometer, the environmental noise from the microphone and the data from the step counter of the device. \bigbreak
In my first experiment, the mobile application gathers the data while the participant is solving a provided programming question. 
Afterwards the user gets asked to answer some additional questions. \\
In a second experiment, a single participants solves Sudokus in a more controlled environment. 
In the two experiments I tried to find patterns in behavior and environment that are influencing the quality of a programmer. The gathered information were clustered into a specific classified behavior or context and being compared in order to find correlations with the code quality.\\
For determining the code quality I used a tool for analyzing the code which has been uploaded on GitHub to calculate a level of quality.
\bigbreak

The individual experiment with only one participant showed some evidence about the influences of two different kinds of music compared to each other and to no music. In this particular case, classical music reached the best results. Caffeine seemed to reduce the cognitive performance of the participant while, on the other hand running before solving the task showed an improvement. 
SOME RESULTS WILL BE MENTIONED

\bigbreak
In the future, the app could become an every-day tool for developers and students. It could be used instead or hand in hand with project-management tools, that require the times how long a programmer worked on a project. It could simultaneously provide real time feedback about the code quality itself or suboptimal aspects in the working environment. Rather than comparing the information with other app users, the app could make use of systematic learning of and optimal working environment and behavior of the specific programmer.

\section{Motivation}
\subsection{Mobile Device Sensors}
Over the last decades the evolution of mobile devices began with a wireless telephone far away from pocket size. Over the years the mobile devices got displays, SMS, telephone books, games and a lot more. In 2007, Steve Jobs introduced the first iPhone and with it the age of the smartphone  \cite{laugesen2010factors}. Over the years, smart-phones became pocket size computers with a better display resolution than the most televisions and the computing power of what desktop pc users could just dream about a few years ago. More and more sensors were packet into the small handy devices were made easy accessible by developers.\\
The range of sensors reaches from proximity detection over accelerometer to humidity sensors etc. Google even engineered a system for 3D objects and indoor environments with just a single device in real time \cite{schops20153d}.\\
So, a lot of people own the hardware with the capability to collect rich context information and they even carry it with them all the time and could be used for support and improve the people's work and environment.

\subsection{Learning to write code}
Learning how to program is getting more and more important and still not a required subject in school. It is the computer with its software which is controlling almost everything in our everyday life such as traffic, gates, calendars etc. It's failure could have dramatic impacts in peoples life.\\
Thus, it is very important that programmers produce high quality code and also be able to find good frameworks and libraries. The problem is that it is not always obvious what high quality means. It can vary from good structured code to resource-aware, reliability and much more. A lot of programmers didn't learn coding in school or university. They taught it to themselves and they might just have used it for fun-project which were not created for public usage. However, what I want to say is that a programmer might not really know how good or bad his/her code really is.\\
I experienced this problem myself.
I started to work in an agency that specialized on iPad apps and design. I was hired because the previous mobile developers left the company and they urgently needed a replacement. When I started I had no practical experience in writing mobile applications. I had to maintain the current code and add new features in a big and unknown project. As I had no mentor or anyone who could give me feedback I just did it as good as I could. I still don't know whether I created good or bad code. With a feedback tool for my code quality I could have learned a lot about the coding itself and by consequence, I would probably write much better code.

\subsection{The importance of Software Metrics}
Software is becoming more complex than ever and used in almost every environment. The deadlines in professional software projects are very strict and there is no time to develop everything from scratch. Development teams depend on libraries rather than reinventing the wheel over and over again. \\
The problem is that it's so easy to start programming and learning everything online from various sources. A lot of people do programming for a hobby and the quality and performance doesn't matter to them as much as in a professional environment. They also create libraries and frameworks with their quality standards. Also the increasing amount of open source libraries and the dependencies they create makes it very hard but also very important to professionals that they can trust these libraries and the quality. At this point of time the only indicators are user ratings and the amount of times it's been used in different frameworks and projects. Some frameworks are also recommended in public reviews articles. \\
Frameworks and libraries also need to be more dynamic and maintained. Operating systems and programming languages are being updated more frequently which requires fast changes. A constant measuring of quality could at the first place give direct feedback that the quality stays constant after changes and as well helps the developer to create a better structure and code to improve the maintainability at the first place. \\
The most used platform in the open source community is Github. Github is based on git and is a web server that can be used to host software projects and allow to make them accessible to others developers  \cite{dabbish2012social}.

\subsection{Working environment}
The feedback of the code quality can then be used to find and improve influencing factors. 
How important is a good working environment?\\
A new trend, especially in the tech industry is going from common clean looking office spaces to colorful creative environments closer to living rooms. Companies like Google or Facebook seem to get rid of the strict separation of work and personal life. Companies introduce unlimited holiday policies, provide free food and even have a laundry service for their employees. They try to remove all the obstacles from their employees life to allow them focus on their work. 
Also the social aspect at work changes a lot. Some years ago, things like having a beer with the co-workers after office hours or meeting the colleagues for a ping-pong match during the day was unthinkable.\\
Google tries to make developers communicate more with the team by placing the whole team in relatively small spaces and provide silent areas for tasks that require more silence. 
All these efforts to make the employees more productive are very interesting approaches but hard to measure. 
\bigbreak
In my dissertation I am trying to find patterns between working environment, behaviors and the resulting code quality in learning and professional environments. I also think that the creation of awareness for code quality and performance is an essential factor in the evolution of a programmer and is important at any stage of the experience and my work could be a base for tools that are providing this information to the software developers and engineers. 

\section{Aims}
I hope to find patterns in the working environment and behavior for in general that significantly influence the quality in software development.
This knowledge might inspire and help future researchers and the industry to do more work in this area and create tools for bringing the code quality of future software to a higher standard. 
Also for academia, a tool that provides feedback on code quality for the students can help to bring them on a higher level when they leave college. 

\section{Road-map}
In the next chapter I am summarizing the state of the art in the area of software metrics, its measurements and analysis, data gathering followed by data clustering and factors that influence cognitive performance and software quality.\\
The following Chapter describes the design of the different software components that are using to gather the data, provide information to the participants and to ensure the privacy of the gathered information. 
Chapter 4 contains information and the process of the implementation of the different software components. 
The experiment is being described in chapter 5 and includes the setup and execution, the expected results and the data classification as well as the questions that are being asked to the participants. 
The last two chapters describe and interpret the results and conclude the subject and information of the experiment

