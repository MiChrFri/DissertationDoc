\chapter{Introduction}

\begin{quote}
\centering 
\em % optional -- to switch to emphasis (italics) mode
"Be a yardstick of quality. Some people aren't used to an environment where excellence is expected."

\medskip
\raggedleft
Steve Jobs
\end{quote}
\vspace{10 mm}

A computer is the powerful hardware that is able to do much more than a human being could be capable to do. However, The enormous capacity is useless without software which is making use of it. In order to make the increasing computations controllable and keep the machines as customizable as possible, different layers of abstractions allow to control and use the computing power. The abstraction reaches from machine readable code over Assembly to high level programming languages which can almost be read as an English sentence and on top of that libraries and frameworks that already provide functionality \cite{Martin:2008:CCH:1388398}. A big part of the complexity is already encapsulated and the software engineers can focus on the functionality of the software and their specific problems.
\bigbreak
Over the years, the performance of the computers rapidly increased and with it the complexity of the code \cite{wirth2008brief}.
Software can be a simple tool that is written in a short time by a single person or it can be a gigantic software project with several hundred developers \cite{cusumano1997microsoft}.
In order to allow to split the work on a software project, an encapsulation of the modules is mandatory. A general structure must be given to ensure that the different parts can integrate hand in hand and to keep the code understandable. The more people work on one project, the more important it is to provide a organized, well planned architecture and keep the code clean.
\bigbreak
In today's world, software is everywhere. The traffic is controlled by computers, security systems, nuclear power plants or just a messenger app on a mobile phone. 
The ubiquity of computers can make life easier, but can also cause unpredictable trouble.\\
In the early 1890s at the United Kingdom's Royal Air Force an engineer found a bug that could fire a missile without any command. \cite{ross2005exterminators}
The quality requirements varies for different software products. A crashing weather app on a mobile is not as bad a bug causing a production stop in a plant. 
However, the quality of the software can make the difference whether a company will be successful or just be one of many abortive start-ups with a good idea but a bad execution.
\bigbreak
In the software industry, the most significant factor in the creating process is the human. The quality strongly depends on the performance of the programmers and that can change by various different reasons even a few times a day. Previous smart people already did research and their experiments and I will start the dissertation with their findings of previous studies that investigated different theories about influencing factors on cognitive performance and related work.
\bigbreak
In this dissertation, my own approach will be to make use of the sensors and information provided by the user's mobile phone. I will write and Android application that the user can install on his/her device, which is gathering the location of the user, collecting sensor data from the light sensor, accelerometer, the environmental noise from the microphone and the data from the step counter of the device. \\
The sensors and the software will gather the data while the participant is working an a provided programming task. Afterwards the user gets asked to answer some questions such as information about disturbance or for example mood and more. \\
Based on that information I will try to find patterns in behavior and environment that are influencing the quality of a programmer. The gathered information will be clustered into a specific classified behavior or context and compare them to find correlations with the code quality.\\
For determining the code quality I will use a tool that can analyze the code that is uploaded on GitHub and calculate a level of quality.
\bigbreak
I expect that this work will find some influencing factors for the general programmer. Of course, programmers are all different and their optimal environment and way to work is different but I hope that this work is providing the basics and shows the importance for giving feedback to software developers in order to improve.
\bigbreak
In the future, the app could become an every-day tool for developers and students. It could be used hand in hand with project management tools, that require the exact times, a programmer worked on a project and it could simultaneously provide real time feedback about the code quality itself or destructive aspects in the working environment. Rather than comparing the information with other app users, the app could make use of systematic learning of and optimal working environment and behavior of the specific programmer. Having this information, the app could inform the user when the optimal environment is not given and provide the information how it could be reached (e.g. "Find a place with better lighting conditions and less environmental noise to increase your productivity"). 


\section{Motivation}
Learning how to program is getting more and more important and still not a requiered subject in school. On the other hand, it is the computer with it's software which is controlling almost everything in our everyday life such as traffic, gates, calendars etc. I failure could have dramatic impacts in peoples life.\\
Therefore it is very important that programmers produce high quality code. The problem is that it is not always obvious what high quality means. It can vary from good structured code to resource-aware, reliability and much more. A lot of programmers didn't learn coding in school or university. They taught it themself and they might just used in for fun-project which were not created for public usage. However, what I want to say is that a programmer doesn't really know how good or bad his/her code really is.\\
I experienced this problem myself.
I started to work in an agency that specialized on iPad apps and design. I was hired because the previous mobile developers left the company and they urgently needed replacement. When I started I had no practical experience in writing mobile applications. I had to maintenance the current code and add new features. As is had no mentor or anyone who could give me feedback I just did it as good as I could. I still don't know whether I created good or bad code. With feedback of the code quality I could have learned a lot lot about the coding itself and by today I would probably write much better code.\\
The feedback of the code quality can then be used to find and improve influencing factors. 
How important is a good working environment?\\
A new trend, especially in the tech industry is going from common clean looking office spaces to colorful creative environments closer to living rooms. Companies like Google or Facebook seem to get rid of the strict separation of work and personal life. Companies introduce unlimited holiday policies, provide free food and even have a laundry service for their employees. They try to remove all the obstacles from their employees life to allow them focus on their work. 
Also the social aspect at work changes a lot. Some years ago, things like having a beer with the co-workers after office hours or meeting for a ping-pong match during the day was unthinkable. 
Google tries to make developers communicate more with the team by placing the whole team in relatively small spaces and provide silent areas for tasks that require more silence. 
All these efforts to make the employees more productive are very interesting approaches but hard to measure. 
\bigbreak
In my dissertation I am trying to find patterns between working environment, behaviors and the resulting code quality in learning and professional environments. I also think that the creation of awareness for code quality and performance is an essential factor in the evolution of a programmer and is important at any stage of the experience and my work could be a base for tools that are providing these information to the software developers and engineers. 

\section{Aims}
I hope to find patterns in the working environment and behavior for in general that significantly influence the quality in software development.
This knowledge might inspire and help future researchers and the industry to do more work in this area and create tools for bringing the code quality of future software to a higher standard. 
Also for academia, a tool that provides feedback on code quality for the students can help to bring them on a higher level when they leave college. 

\section{Road-map}
In the next chapter I am summarizing the state of the art in the area in software metrics, its measurements and analysis, data gathering followed by data clustering and factors that influence cognitive performance and software quality. 
Chapter 3 explains why it is the right time for this project and why it is important. 
The following Chapter describes the design of the different software components that are using to gather the data, provide information to the participants and to ensure the privacy of the gathered information. 
Chapter 5 contains information and the process of the implementation of the different software components. 
The experiment is being described in chapter 6 and includes the setup and execution, the expected results and the data classification as well as the questions that are being asked to the participants. 
The last two chapters describe and interpret the results and conclude the subject and information of the experiment

