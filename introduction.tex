\chapter{Introduction}

\begin{quote}
\centering 
\em % optional -- to switch to emphasis (italics) mode
"Be a yardstick of quality. Some people aren't used to an environment where excellence is expected."

\medskip
\raggedleft
Steve Jobs
\end{quote}
\vspace{10 mm}

Over the last years, the performance of computers rapidly increased and with it the complexity of the Software \cite{wirth2008brief}.
It can be a simple tool that is written in a short time by a single person or it can be a gigantic software project with several hundred developers involved in it\cite{cusumano1997microsoft}.
Different layers of abstraction from low level to high level programming languages actually make it possible to reduce the complexity of software projects. On top of that, there are libraries and frameworks that provide features that already have been implemented \cite{Martin:2008:CCH:1388398}.
In order to allow splitting the work in a software project, an encapsulation of the modules is mandatory. A general structure must be given to ensure that the different parts can integrate easily and to keep the code understandable.\\
The more people work on one project, the more important it is to provide an organized and well planned architecture to keep the code clean.
\bigbreak
In today's world, software is everywhere; the traffic is controlled by computers as well as security systems, nuclear power plants or just a messenger app on a mobile phone etc. \\
The ubiquity of computers can make life easier, but can also cause unpredictable trouble.\\
In the early 1890s at the United Kingdom's Royal Air Force, an engineer found a bug that could have fired a missile without any command. Luckily it was found before a disaster happened \cite{ross2005exterminators}.
\bigbreak
The quality requirements vary for different software products. A crashing weather app on a mobile is not as bad as a bug that is causing a production stop in a plant. 
However, the quality of the software can make the difference whether a company will be successful or just be one of many abortive start-ups with a good idea but a bad execution.
\\
In the software industry, the most significant factor in the creating process is the human. The quality strongly depends on the performance of the programmers as a single person or in a team. That performance quality can change by various different reasons even a few times a day.
\bigbreak 
Previous researchers already did a lot of work in the field. \\
This dissertation will start with an overview of their findings which will include previous studies that investigated different theories about influencing factors on cognitive performance and related work.

\bigbreak
This dissertation describes a new approach where sensors and information provided by the user's mobile phone were used to find evidence in factors that influence the performance of developers. 
We developed an Android application for being installed on the participants mobile device. The app is gathering the location of the user, collecting sensor data from the light sensor, accelerometer, the environmental noise from the microphone and the data from the step counter of the device. \bigbreak
In the first experiment, the mobile application gathers the data while the participant is solving a provided programming task and afterwards answering additional questions. \\
In a second experiment, only one participant solves Sudokus in a more controlled environment. 
Two experiments investigated patterns in behavior and environment that are influencing the cognitive quality of a programmer. The gathered information were investigated to find correlations with the code quality.\\
\bigbreak

The individual experiment with the single participant provided evidence about influences of different genres of music compared to each other. Both resulting performance were also compared to a control scenario with no music played. In this particular case, classical music reached the best results compared to heavy metal and the absence of music.\\
Also caffeine seemed to reduce the cognitive performance of the participant while, on the other hand running before solving the task showed evidence of a positive effect.
The results from the experiment with a group of participants shows no distinct patterns but traces of correlations between strong changes in the environment and distraction of the participant.  

\bigbreak
In the future a monitoring of developers performance could become a ubiquitous tool for developers and students. It could be used hand in hand with project-management tools, that require the duration how much time a programmer worked on a specific project. It could simultaneously provide real time feedback about the code quality itself or divergent aspects in the working place and work patterns. Rather than only comparing the information with other app users, the app could also systematically develop an optimal working environment and behavior of the specific programmer.

\section{Motivation}
This section will describe the factor that motivated the topic of this dissertation and some background information. 

\subsection{Mobile Device Sensors}
Over the last decades the evolution of mobile devices began with a wireless telephone far away from pocket size. Over the years, the mobile devices got displays, SMS, telephone books, games and a lot more. In 2007, Steve Jobs introduced the first iPhone and with it the age of the smartphone  \cite{laugesen2010factors}. Over the years, smart-phones became powerful computers with a better display resolution than the most televisions and the computing power of what desktop pc users could just dream about a few years ago. More and more sensors were packet into the small handy devices.\\
The range of sensors reaches from proximity detection over accelerometer to humidity sensors etc. Google even engineered a system for 3D objects and indoor environments with just a single device in real time \cite{schops20153d}.\\
Thus, a lot of people own the hardware with the capability to collect rich context information and they even carry it with them all the time and has a lot of potential to support and improve the people's work and environment.

\subsection{Learning to write code}
Learning to write code is getting more and more important but still not a required subject in school. It is the software that is controlling almost everything in our everyday life such as traffic, gates, calendars etc. It's failure could have dramatic impacts in peoples life.\\
Thus, it is very important that programmers produce high quality code and also be able to find good frameworks and libraries. The problem is that it is not always obvious what high quality means. It can vary from good structured code to resource-aware, reliability and much more.
\\Still, many programmers didn't learn coding in school or university. They taught it to themselves and might just have used it for fun-project which were not created for public usage. However, they might not really know about best practices in the industry or how good or bad his/her code really is.\\
I experienced this problem myself.
When I started my first job as an iOS developer, I had no practical experience in writing code mobile applications. I had to maintain the current code and add new features in a big and unknown project. As I had no mentor or anyone who could give me feedback, I just did it as good as I could. I still don't know whether I created good or bad code. With a feedback tool for my code quality I could have learned a lot about the coding itself and by consequence, I would probably write much better code.

\subsection{The importance of Software Metrics}
Software is becoming more complex than ever and used in almost every environment. The deadlines in professional software projects are very strict and there is no time to develop everything from scratch. Development teams depend on libraries rather than reinventing the wheel over and over again. \\
As already mentioned in the previous section, a lot of people do programming for a hobby where the quality and performance doesn't matter to them as much as in a professional environment. They also create libraries and frameworks with their quality standards. The increasing amount of open source libraries and dependencies make it very hard, but also very important to professional developers to be able to trust the quality and functionality. At this point of time, the only indicators are user ratings and the amount of times it's been used in different frameworks and projects. Some few frameworks are also recommended in public reviews articles. \\
Frameworks and libraries also need to be dynamic and maintained. Operating systems and programming languages are being updated more frequently which requires fast changes. A constant measuring of quality at the first place could give direct feedback that the quality stays constant after changes and as well helps the developer to create a better structure and code to improve the maintainability from the beginning. \\
The most used platform in the open source community is Github. Github is based on git and is a web server that can be used to host software projects and allow to make them accessible to others developers  \cite{dabbish2012social}.

\subsection{Working environment}
A new trend, especially in the tech industry shows that company move from common clean looking office spaces to colorful creative environments. Companies like Google or Facebook seem to reduce the strict separation of work and personal life. Companies introduce unlimited holiday policies, provide free food and even have a laundry service for their employees. They try to remove all the obstacles from their employees life to allow them focus on their work.\\
Also the social aspect at work changes a lot. Some years ago, having a beer with the co-workers after office hours or meeting the colleagues for a ping-pong match during the day was unthinkable.\\
Google tries to motivate developers to communicate more with the team by placing the whole team in relatively small spaces and provide separate areas for tasks that require more silence. 
All these efforts to make the employees more productive are very interesting approaches but hard to measure. 
\bigbreak
In this dissertation, we are trying to find patterns between working environment, behaviors and the resulting code quality in learning and professional environments. Also, the creation of awareness for code quality and performance is an essential factor in the evolution of a programmer and is important at any stage of the experience. 

\section{Aims}
The goal is create a solid base and a working demonstration of a system that can gather data from a mobile device and show the significance of influences the quality in software development. 
This work and framework hopefully inspires and helps others to do more work in this area to bring code quality to a higher standard. 
Also for academia, a tool that provides feedback on code quality for the students can help to bring them on a higher level when they leave college. 

\section{Road-map}
The next chapter summarizes the related work in the area of software metrics, measurements and analysis, data gathering, data clustering, influences in cognitive performance and software quality.\\
Ensuing, we describe the design of the different software components to gather the data, provide information to the participants and to ensure the privacy of the gathered information.\\
Chapter 4 contains information and the process of the implementation of the different software components. 
The experiment is being described in chapter 5 and includes the setup and execution, the expected results. the next chapter characterizes data classification as well as the questions that are being asked to the participants. \\
The last two chapters describe and interpret the results and conclude the subject and information of the experiment.
The appendix contains an overview of the abbreviations, links to the source code and more details about results of the experiments.

