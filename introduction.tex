\chapter{Introduction}

\begin{quote}
\centering 
\em % optional -- to switch to emphasis (italics) mode
"Be a yardstick of quality. Some people aren't used to an environment where excellence is expected."

\medskip
\raggedleft
Steve Jobs
\end{quote}
\vspace*{\fill}

\begin{flushleft}
Since the beginning of the computer, software was needed to be written. Software can be a simple tool that is written in a short time by a single person or it can be a gigantic software project with several hundred developers \cite{cusumano1997microsoft}.
Today, software is everywhere. The traffic is controlled by computers, security systems, nuclear power plants or just a messenger app on a mobile phone. 
The ubiquity of computers can make life easier, but can also cause unpredictable trouble.

In the early 1890s at the United Kingdom's Royal Air Force an engineer found a bug that could fire a missile without any command. \cite{ross2005exterminators}
The quality requirements varies for different software products. A crashing weather app on a mobile is not as bad a bug causing a production stop in a plant. 
However, the quality of the software can be significant whether a company will be successful or just be one of many abortive start-ups with a good idea but a bad implementation. 

In the software industry, the most significant factor in the creating process is the human. The quality strongly depends on the performance of the programmers. 
People's performance can change by various different reasons. In this dissertation I am trying to find correlations between the behavior and environment with the performance and creativity of programmers and computer science students. 

Using the sensors and information provided by the user's mobile phone, I am clustering it into a specific behavior or context and search for correlations with the code quality. 

\end{flushleft}



\section{Motivation}

\section{Aims}

\section{Road-map}
	Chapter 2 - State of the Art
	Chapter 3 - Design of Experiments 
	Chapter 4 - Implementation
	Chapter 5 - Evaluation
	Chapter 6 - Conclusion 

