\chapter{Implementation}

\begin{flushleft}

\end{flushleft}

\section{Android}
As mentioned before, the Android Application is for sensoring data of the user and get environmental information. For This purpose I wrote an Android application which can gather these information. 
Beside gathering the data using an App it is also possible to read the sensoring information which are being recorded continuously as described in the approach of
Zhu, Hengshu, et al. \cite{zhu2015mining}. They are reading the device logs and get all the logged device more information about the apps being used etc. .
Sandboxing is an Android security concept that only allows an app to access the data of the app itself and isolates the content for other applications. Thus it is impossible to access the device logs via an app without having physical access to the device. 
In terms of the ideas for future usage of the app it doesn't make sense to require physical access to the device itself. Thus, the decision to use an App, installed on the users device, is the best way to go for this purpose.

The implementation of the Android application has been done using the Android Studio IDE, which is provided free usage by Google, Inc. The code was written in Java, which is the official programming language for Android applications. Google also provides a variety of libraries and frameworks for user interface-Elements and basic functionality. For the user interface Android Studio has build in Solutions to either design the graphical user interface (GUI) using Java code or defining the elements in XML files. 
The  


User ID is been generated by creating a SHA256 hash value of the email address. 

\subsection{Data Gathering}
The app is gathering the data every few seconds, between every 2 and 10 seconds, depending on the device speed. After getting all the values from the sensors, microphone and Android OS, the app is generating a timestamp, adds the user ID to the 


\subsection{Data Storage}
The gathered information are being combined to one entry for each collected timestamp. 

\subsection{User Interface}

\subsection{Security}
The user can't be identified by the user id because it is been generated by a SHA256 hash function that is infeasible to invert. In other words, the SHA256 algorithm generates a base16-String from the email-address of the user and there is no mathematical known way to recover the original email address in feasible time from the base16-String. 




Hybrid cryptography using AES and RSA. RSA with the public key of a pre-generated 1024 bit key-pair. 

The en and decryption uses AES with CBC and an PKCS5Padding. 



AES generates a 128 bit key with a random SHA1 seed every time the app restarts. 


