\chapter{Conclusions}

And a fancy conclusion...


\section{Future Work}
The results give evidence that it is very hard to find general patterns in the influences in the cognitive performance. The individual results were more clear but just valid for a single person. 
In my opinion it would be a good idea to let people individually gather data about their environment and working patterns continuously. A machine learning algorithm on the phone could find correlations between the individual performance and the data. Also, as the first experiment was just focusing on one single task, it would make sense to send the gathered data of each user to a Server that is analyzing it in relation to the data sets of the other participants. 
In experiment one, some information from some participants were not gathered because of the permissions and settings of their individual mobile devices or in some cases probably even missing sensors that were not built in that device. 
Therefore, to be able to eliminate the hardware issues a custom device to place on the desk would help to avoid these issues. It would also allow to add more sensors such as temperature, humidity etc. and with the same hardware guarantee that the sensors are working equally. 
Another idea would be to make use of wearables such as smart-watches, fitness trackers or medial devices for monitoring the body functionality. Rather then gathering data from the environment here the body would stay in the focus. The cognitive performance could for example be correlated to the oxygen level in the blood or some sensors that are not on the market, yet. 

More work in the research field is also needed as by now the most influences in cognitive performance are just discovered in experiments reasoned with theories but rarely scientific facts. In order to find more influences, there is deeper knowledge necessary in understanding the human brain and cognition. 

