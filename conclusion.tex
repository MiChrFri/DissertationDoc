\chapter{Conclusions}
This chapter summarizes the findings of this dissertation and the evidence that have been accumulated. This chapter concludes the work and results as well as the contribution to the research area. 
The chapter will be completed with a direction for future work in this area and the potential of research in this field. 

\section{Project Overview}
This dissertation investigated correlations between temporarily environmental influences and human behavior in their cognitive performance. 
An Android application was used by participants to gather data about light, volume, location, accelerator and the step-counter with a timestamp.  
Two different experiments, one with a single participant and another one with a group of subjects, were executed. The first experiment gathered data while the participants worked on a given programming task. In the second experiment, I investigated isolated factors against each other.
The first experiment does not provide clear results. It provides a plausibility that distractions of the participant can be correlated to a dynamic light level in in the environment. 
The results from the second experiment give more clarity and show evidence of a negative influence by a high dose of caffeine as well as listening heavy metal music while working. On the other hand, there are indicators that classical music while working has a positive effect on the cognitive performance  as well as high physical activity before the tasks. 

\section{Contribution}
The dissertation describes the development of a system for finding influences in the software development. The results show that the approach with an application for gathering data and afterwards comparing with analyzed code worked well and has the potential to find evidence in influencing factors for a larger scope.
The source code of the Android app and all the tools from dissertation are available on Github for unrestricted use. The links to the repositories can be found in the appendix. 
\bigbreak
This dissertation portrays the first steps for measuring the environment and the behavior of the developer. It distinguished the data into context and compared it with other results in the area of cognitive performance and software development. A lot of previous work is about long term effects in cognitive performance while this dissertation investigated the instant influences who are actually easier to change by the developers themselves.\\
This work demonstrates the possibility to correlate code metrics with it's changing influences that could be used to optimize the processes and environments in software development. 

\section{Future Work}
In order to find evidence for influencing factors in software development it is necessary to work with much more participants who produce more source code.
\bigbreak
Also, as already mentioned in the Design chapter, this dissertation provides a good base for create an ecosystem for consonantly provide a develop with feedback about the quality of code and environment. That system could provide real time feedback to the developer or project managers and give suggestions who to improve the work based on collected analyzed data. 
Also wearables such as smart-watches, fitness trackers or medical devices are entering the market and are also providing even more information about the developers fitness, health and possibly more. 
Long term studies with more participants would also help to create more accurate results and with that knowledge help the software industry to improve the quality and help future developers to be aware about their code quality. 
More work in the research field is also needed. As by now, many influences in cognitive performance are only discovered in experiments reasoned with theories but rarely scientific facts. In order to find more influences, a deeper knowledge is necessary to fully understand the human brain and cognition. 

