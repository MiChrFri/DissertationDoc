\chapter{Conclusions}
In this chapter, I write about the finding of this dissertation and the evidence that have been accumulated. This chapter concludes the work and results as well as the contribution to the research area. 
The chapter will be completed with a direction for future work in this area and the potential of this research field. 

\section{Project Overview}
This dissertation investigated correlations between temporarily environmental influences and human behavior in their cognitive performance. 
An Android application was used by participants to gather data about light, volume, location, accelerator and the step-counter with a timetamp.  
Two different experiments, one with a single participant and another one with a group of subjects, were executed. The first experiment gathered data while the participants worked on a given programming task. In the second experiment, I investigated isolated factors against each other.
The results from the first experiment don't show very clear results. It provides a plausibility that distractions of the participant can be correlated to string changing environments such as changing lighting conditions or sudden noises. 
The results from the second experiment give more clarity and show evidence of a negative influence by a high dose of caffeine as well as listening heavy metal music while working. On the other hand, there are indicators that classical music while working has a positive effect on the cognitive performance  as well as high physical activity before the tasks. 

\section{Contribution}
This dissertation describes a new approach for measuring the environment and the behavior of the developer. It distinguished the data into context and compared it with other results in the area of cognitive performance and software development. A lot of previous work is about long term effects in cognitive performance while this dissertation investigated the instant influences who are actually easier to change by the developers themselves. \\
This word demonstrates the possibility to correlate code metrics with it's changing influences which could be used to optimize the processes and environments. 

\section{Future Work}
This dissertation provides a good base and approaches to create applications that can provide the software developers with real time feedback about their code quality and also constantly monitor the environment and learn from it. With learning algorithms, researchers or developers could create a system that can find factors in environment and behavior that influence the quality of the development process and the code by itself. That system could provide real time feedback to the developer or project managers and give suggestions who to improve the work based on collected analyzed data. 
Also, as the second experiment in this dissertation was focusing on one single task, it would make sense to send the gathered data of each user to a Server that is analyzing it in relation to the data sets of the other participants. 
In experiment one, information from some participants were not gathered because of the permissions and settings of their individual mobile devices or in some cases probably even missing sensors that were not built in that device. 
Therefore, to be able to eliminate the hardware issues a custom device to place on the desk would help to avoid these issues. It would also allow to add more sensors such as temperature, humidity etc. and with the same hardware guarantee that the sensors are working equally. 
Another idea would be to make use of wearables such as smart-watches, fitness trackers or medial devices for monitoring the body functionality. Rather then gathering data from the environment here the body would stay in the focus. The cognitive performance could for example be correlated to the oxygen level in the blood or some sensors that are not on the market, yet. 

More work in the research field is also needed as by now the most influences in cognitive performance are just discovered in experiments reasoned with theories but rarely scientific facts. In order to find more influences, there is deeper knowledge necessary in understanding the human brain and cognition. 

